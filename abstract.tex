\begin{abstract} Smartphones represent the most serious threat to user privacy
of any widely-deployed computing technology because these devices are always on
and always connected, making them the perfect candidate to know most about the
owner. Unfortunately, existing permission models provide smartphone users with
limited protection, in part due to the difficulty to users in distinguishing
between legitimate and illegitimate use their data; for example, a mapping app
may upload the same location information it uses to download maps (legitimate)
to a marketing agency interested in delivering location-based ads
(illegitimate). As a result, smartphone users find themselves forced to make
burdensome and error-prone tradeoffs between app functionality and privacy. To
combat this, we propose a new approach, called DataDecoy. By allowing
substitution of real data streams with artificial or \textit{mocked} data,
DataDecoy allows users to manipulate impressions of their behavior in
well-defined ways: to appear more fit, more social, or more on-time than they
actually are. Instead of focusing on privacy, we explore providing users with
better management of their smartphone-derived digital identities. We discuss
the design of DataDecoy, which uses user-initiated context trace recording and
replay to enable objective-driven context mocking. Our evaluation shows that
users want to use DataDecoy, that DataDecoy can mock popular smartphone apps,
and that DataDecoy is usable.  \end{abstract}
