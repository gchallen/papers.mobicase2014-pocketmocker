\section{Future Work}
\label{sec-future}

By enabling objective-driven context mocking, \PocketMocker{} opens up many
new directions for future work.

\subsection{Improving security and user experience}

While we have demonstrated that \PocketMocker{} can effectively mock
unsuspecting smartphone apps, its security and user experience can still be
further improved. Because of its implementation at the platform-level, there
are possibilities for apps to know when they are being mocked. For example,
they can read data directly from the kernel instead of calling through the
Android platform. Our next step is to better protect \PocketMocker{} by making
changes at the kernel level. Another flaw with the current prototype is that
mocking happens globally, meaning all user-installed apps are mocked when
replaying mocked data. This can cause a problem if a user wants to use a
user-installed navigator, but still want to mock another app while driving. We
are working towards user-specified app mocking, so only apps selected by the
user are mocked.

\subsection{White hat data analytics}

Currently the process of linking user objectives with mocking traces is
qualitative in nature: \PocketMocker{} can suggest that a user interested in
seeming more fit take a walk. We want to be able to provide users quantitative
information as well, such as how apps are using their behavioral data to
classify them. We will be augmenting \PocketMocker{} with a library of
open-source data analysis algorithms, so users can see how their data might be
used to determine who they are. These open-source algorithms---or \textit{white
hat data analytics}---can also be used to help a user find out how well their
mocking traces are working.

\subsection{Sharing mocking traces}

\PocketMocker{} is also limited by the device owner's previously recorded
actions. While this is a powerful primitive, it is also limited, including in
ways that complicate \PocketMocker{}'s task of preserving spatial continuity as
described previously. We can grow their dataset in the background as they go
about their daily routine by performing location-driven and battery-permitting
recordings. There is also potential for users to share mocking traces, which
would allow users to replay activity that they never performed, thus providing
them even more options to shape their digital identity.

\subsection{Mocking to test}

Finally, mocking can be used as a developer tool. By adding mocking support to
Android, developers can test in a far more realistic way than ever before. For
example, a developer building a new navigator would be able to provide all
required sensor data and see how the app performs with real data while still
sitting at a desk.
