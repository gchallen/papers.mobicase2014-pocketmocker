\section{Future Work}
\label{sec-future}

By enabling objective-driven context mocking, \PocketMocker{} opens up many
new directions for future work.

\subsection{Securing the Mocking Context}

While we have demonstrated that \PocketMocker{} can effectively mock
unsuspecting smartphone apps, future apps may attempt to defeat user mocking
by performing more validation of the information they receive from smartphone
platforms. As described previously in Section~\ref{sec-implementation},
because it is implemented at the platform layer our current prototype cannot
defend against even simple attacks that bypass Android and read data directly
from Linux. Our next step is develop a set of mocking patches allowing
mocking to be implemented in Linux, which will address this issue.

More work is needed, however, to determine the best approaches to preventing
apps from observing differences between the mocked context and the real
environment. As an example, while it is simple to make a faster network
device (Wifi) look like a slower one (3G) through bandwidth throttling, the
reverse is not possible. However, one approach may be to mock conditions
associated with a congested or low-signal strength Wifi link when the real
context is using 3G, or mislead the app into thinking that its bandwidth is
being throttled by the Android platform.

\subsection{Per-App Mocking}

At the moment \PocketMocker{} feeds mocked data to all apps, which can create
undesirable consequences in certain cases. As an example, if a user wants to
perform mocking but also use their smartphone for navigation during the
mocking session, this is currently impossible since the navigation app will
be receiving incorrect locations from \PocketMocker{}. We are working on
improving the platform support \PocketMocker{} requires to allow
user-specified apps to pierce the mocking context and receive real data. Once
able to perform mocking on a per-app basis, \PocketMocker{} could mock
several apps simultaneously using different mocking traces designed to
achieve different objectives.

\subsection{White Hat Data Analytics}

Currently the process of linking user objectives with mocking traces is
qualitative in nature: \PocketMocker{} can suggest that a user interested in
seeming more fit take a walk. To make this process more quantitative, we are
augmenting \PocketMocker{} with a library of data analysis algorithms that
attempt to replicate the approaches taken by companies mining data, such as
location traces, provided by smartphone apps. We refer to these
open-source algorithms as \textit{white hat data analytics}, reflecting their
intention to help individual users in the name of privacy, rather than
companies in the name of profit. Running these algorithms on the phone,
rather than in the cloud, preserves users' privacy. White hat data analytics
will play two roles in future versions of \PocketMocker{}. First, they will
reveal to users what their smartphone data might reveal about them helping
them choose what attributes to mock. Second, they provide a way of
quantitative evaluating mocking traces, allowing Bob to measure the impact
his walk will have on his activity level and compare it with, for example, a
bike ride.

\subsection{Mocking Libraries and Synthesis}

Mocking performed by \PocketMocker{} is currently limited to repeating
actions that the user has previously recorded. While this is a powerful
primitive, it is also limited, including in ways that complicate
\PocketMocker{}'s task of preserving spatial continuity as described
previously. However, as the size of \PocketMocker{}'s trace library grows it
becomes more likely that our system \textit{can} synthesize new mocking
activities by combining parts of earlier traces. Future versions of
\PocketMocker{} will facilitate this process by performing location-driven
and battery-permitting background recording of context traces as users go
about their daily lives. It is also possible that \PocketMocker{} could
eventually draw on shared libraries of context information to synthesize
activities even falling outside of a user's own library. If another
\PocketMocker{} user has mapped the walking paths, Wifi locations, visible
cell towers, and other shared context information at a particular location,
then \PocketMocker{} may be able to merge that data into an individual trace
of a user walking to allow them to take a walk in a place they have never
visited.

\subsection{Mock to Test}

Finally, we also note that mocking traces may also be valuable to smartphone
developers interested in testing how their app performs under varying
conditions. By adding support to the Android emulator for reading mocking
traces, the user of a new mapping client could experience exactly how their
app would perform for another user navigating the streets of a city far away.
This feature is similar to Android's current ability to allow apps to use
mocked locations, but by also mocking all other location-varying aspects of
the device's context \PocketMocker{} would provide more realism.
