\begin{figure}[t]
\centering
\includegraphics[width=0.8\textwidth]{./figures/architecture.pdf}

\caption{\textbf{\PocketMocker{} Design.} Details are specific to Android but
would be similar on other platforms.}

\label{fig-design}
\vspace*{-0.2in}
\end{figure}

\section{\PocketMocker{} Design}
\label{sec-design}

This section describes the design of the \PocketMocker{} objective-driven
context mocking system. We begin by developing a set of design requirements
based on the mocking scenarios and taxonomy presented previously and outline
the challenges of effective mocking. We then describe how \PocketMocker{}
uses changes to the smartphone platform and a dedicated app to perform
context mocking.

\subsection{Overview}

We offer the example of Bob from our earlier scenarios as an overview
of how \PocketMocker{}'s components work together to deliver objective-driven
context mocking. \PocketMocker{} consists of two parts:  modifications to the
smartphone platform needed to record and replay mocking traces, and an app
that interacts with the user to record mocking traces and control the mocking
process. Figure~\ref{fig-design} illustrates the interaction between the two
parts of \PocketMocker{}.

Bob knows his objective: to appear more active. All he has to do now is collect
some sample data so \PocketMocker{} can inject mocked data to the apps on his
smartphone when he wants to be active. First, he must record a mocking trace of
his desired outcome---in this case, the action is taking a walk. During this
phase, \PocketMocker{} records all sensor data---such as GPS, cell tower
metadata, visible WiFi access points and more---on the device to provide a
concrete mocking context. All of this data combined is known as the \textit{mocking trace}.

Once the trace is recorded, Bob can replay it as often as he likes. During the
mocking process, \PocketMocker{} exploits changes to the underlying
smartphone platform to satisfy app requests for real data with time-shifted
false numbers from the mocking trace. While the mocking session is active, the
\PocketMocker{} app displays a notification indicating that mocking is in
progress and how much time remains before it finishes. After completion,
\PocketMocker{} stops returning mocked data and resumes returning real data to
apps.

Based on this use case, we can enumerate several requirements for an
objective-driven context mocking system. First, the user's objective must be
determined and a mocking activity suggested. Second, a mocking trace must be
recorded and linked to the user's objective. Finally, the trace must be
deployed as needed to achieve the user's mocking objective. We describe how
\PocketMocker{} accomplishes these tasks and overcomes two consistency
challenges below.

\subsubsection{Linking mocking traces and objectives:\space}

To begin, \PocketMocker{} must be able to link mocking traces with
user-defined objectives, so that it knows what trace will achieve each
objective. In Bob's example, \PocketMocker{} must be able to associate the
trace of Bob taking a walk with Bob's desire to appear more active. This
process has both a qualitative and quantitative component. Qualitatively,
\PocketMocker{} may suggest activities that would naturally be linked with a
specific objective. If Bob wants to be more active, he should take a walk. If
Alice wants to appear more healthy, she should eat at a healthy
restaurant---and not eat at the fast-food restaurant. \PocketMocker{}'s app
provides a library of objectives (''appear more fit'') and associated
suggestions for mocking traces (''take a walk''). We also expect users will
be well-served by their own intuition.

\subsubsection{Collecting, storing, and replaying traces:\space}

Second, \PocketMocker{} must be able to collect, store and replay mocking
traces. Trace collection is currently initiated by the \PocketMocker{} app and
performed entirely at the app level. To ensure that during mocking
\PocketMocker{} can return data from any source consistent with the mocked
context, \PocketMocker{} currently enables all sensors that could provide
relevant information and samples them aggressively, storing timestamped data in
a set of local databases. Unlike trace collection, mocking trace replay
requires platform support.  \PocketMocker{} modifies the underlying smartphone
platform to add an interface allowing it to inject mocked data. Once the user
begins replaying the trace, \PocketMocker{} reads data from all sensor
databases associated with the trace and uses this new interface to inject it
into the platform.  Any app requests for data contained in the mocking then
return mocked data.

\subsubsection{Initiating mocking sessions:\space}

Third, \PocketMocker{}'s app helps the user remember to initiate mocking
sessions. Each recorded trace can be annotated with a frequency which
\PocketMocker{} uses to help prompt the user to deploy the trace. For
example, Bob may want to go for a walk daily, and by annotating the trace
with his goal \PocketMocker{} knows when to provide reminders.

\subsection{Consistency Challenges}

A significant challenge when mocking is addressing differences between the
mocking context and the real context to ensure that mocking proceeds
consistently. At present \PocketMocker{} does not attempt to fully defend the
mocking context from suspicious apps---we leave that challenge as future
work. However, \PocketMocker{} still attempts to ensure that the mocking
context is consistent and does not create obvious problems or physical
impossibilities that could either break app functionality or send an
unmistakable signal that something unusual is happening. First, we look at
how \PocketMocker{} masks differences between the mocking context and the
real context. Second, we address spatial continuity, a specific consistency
problem facing the \PocketMocker{} system.

\subsubsection{Differences with the mocking context:\space}

Here we examine specific differences between the mocking context and the real
context and address how \PocketMocker{} deals with each case:

\begin{itemize}

\item \textbf{Location:} \uline{the phone is one place in the mocking context
and another location in the real context.} To ensure location consistency,
\PocketMocker{} collects all data associated with the mocked location. During
the mocking session an app will not only have the mocked location coordinates
returned, but will also see the same Wifi access points and be connected to
the same cell tower with the same signal strength as it would at the mocked
location.

\item \textbf{Device configuration:} \uline{the accelerometer was not used
during the mocking context but is enabled by an app in the real context.}
Here \PocketMocker{} exploits the fact that it records all information about
the smartphone while recording the mocking trace, meaning that it can handle
requests to use any device feature during replay.

\item \textbf{Connectivity:} \uline{the phone was connected during the
mocking context but there is a different or no network connection available
in the real context.} There are two cases to consider here. If the smartphone
has any connection in the real context, \PocketMocker{} will allow apps to
use that connection but return mocked connection \textit{attributes}. So if
the connection is actually over a Wifi network but only a 3G mobile data
network is available, \PocketMocker{} will establish connections over the
available network but tell apps that they are connected over the mocked Wifi
network. At present \PocketMocker{} makes no attempt to alter connection
properties such as latency or bandwidth of the real connection to match the
mocked connection, and in some cases this is not possible. We leave dealing
with attempts by suspicious apps to use these properties to pierce the mocking
context as future work. There is also the problem of providing a mocked
connection when no real connection exists. Even though this is impossible to
practically accomplish, we can synchronize with the real context have
\PocketMocker{} simply return that there is no active data connection because
networks naturally come and go, so this will not look suspicious to
applications that attempt to circumvent the mocking context.

\end{itemize}

It is also important to point out that \PocketMocker{} \textit{does
not mock}: user interaction, battery level and the microphone
readings. While mocking user interaction may be necessary to mislead certain
types of apps, it is not necessary to mock the apps that \PocketMocker{}
currently targets that collect and interpret data collected passively. More
importantly, replaying interaction would prevent the user from using their
smartphone while mocking was active. Mocking battery levels represents another
continuity challenge; at the time of entering or leaving the mocking context,
the battery level will significantly increase or decrease to match the real or
mocking context and apps can use this jump to detect a context switch has
occurred, thus exposing a hole in the mocking context. Another problem with
mocking battery levels is it creates a confusing user experience because the
user then does not know the current status of the smartphone, which could have
much harsher consequences, so \PocketMocker{} uses the real battery levels
always. Like battery levels, microphone readings also represent a continuity
challenge which leaves a hole in the mocking context and we plan on addressing
this current limitation of \PocketMocker{} in the future.

\subsubsection{Ensuring spatial continuity:\space}

Given that smartphones can and do track their users location, and that this
information reveals a great deal about their lives, \PocketMocker{} is
designed to allow mocking location and user movement. However, this creates a
continuity challenge when the mocking trace ends at a different place from
the user's current location. We discuss in Section~\ref{sec-future} how
future version of \PocketMocker{} will use user-generated mocking libraries
to be able to synthesize mocking traces linking any two points where the user
has previously been, but our current prototype has no good way to address
this problem. And while we are currently focusing on mocking unsuspecting
apps and not addressing all attacks apps could perform on the mocking
context, sudden changes in location are both an all-too-obvious indication of
mocking and might also cause some apps to malfunction.

Currently, \PocketMocker{} works around this challenge by interacting with
the user. While a mocking trace is being replayed, a notification is
displayed informing the user of the time left before the mocking process
completes and the distance from the user to the location where the trace
ends. Once the trace ends, if the user's location is close to where the trace
completes \PocketMocker{} will simply allow the trace to end normally and
merge the real and mocking context. If the user is not close to where the
trace completes, \PocketMocker{} generates an notification asking the user to
either reach the correct location or allow the spatial discontinuity. Until
the user responds to the dialog or reaches the required location,
\PocketMocker{} continues to mock them at the mocking traces final location,
using the lingering capability described next. This also allows
\PocketMocker{} to perform backward time shifting on an existing trace. In
Jerry's example, when it is time to return home he initiates a pre-recorded
trace of his return. Once his trace reaches home, it will linger there until
he arrives.

\subsection{Lingering}

In addition to the record-and-replay functionality we have already described,
\PocketMocker{} also supports \textit{lingering}. Lingering is a mocking
primitive that can be used in several ways: to time-extend a mocking trace,
to conceal an undesirable activity, or to perform time-shift mocking. In the
scenarios described earlier both Alice, Jerry, and Carol's mocking activities
require this capability. Carol uses lingering to conceal her coffee break,
Jerry uses it to time-shift his return home, and Alice uses it to time-extend
her visit to the healthy restaurant to match the time she spends eating fast
food.

To linger, \PocketMocker{} records a small amount of context at a particular
location and then replays it repeatedly. To make the data delivered to apps
during the lingering process more realistic, and prevent apps from detecting
the mocking process by observing repeated readings, \PocketMocker{} injects
noise into the data returned during the lingering session, performing small
changes to the reported location, sensor readings, scan results, and signal
strengths.

To time-extend a trace, the app allows users to indicate linger points during
trace recording. During replay, once the trace reaches a linger point
\PocketMocker{} will linger until instructed to proceed by the user. Once
Alice is ready to leave the fast-food restaurant, she tells \PocketMocker{}
to proceed past the linger point she inserted into her trace of visiting the
healthy restaurant. To conceal an undesirable activity, the \PocketMocker{}
app allows lingering to be initiated at any time. A small amount of context
is recorded and then replayed until the lingering session is canceled. So
Carol can initiate the lingering session at work, and then bring her phone to
her coffee break while appearing to remain at work.

Forward-shifting a trace is a three-step process. First, \PocketMocker{}
collects a small amount of context at the current location in order to linger
and begins the lingering process. Second, \PocketMocker{} records a user
moving to a new location while continuing to return lingering data. Finally,
once the user is ready to merge their real and mocked context,
\PocketMocker{} stops lingering and begins replaying the transition trace
until the user reaches their current location.
