% 25 Mar 2014 : GWA : Goal is 1.5 pages including scenarios and a taxonomy of
% different types of mocking.

\section{Motivation}
\label{sec-motivation}

Recent studies show that privacy is one of today's smartphone users' top
concerns with their devices, second only to battery
life~\cite{truste-privacy}. A plurality of 43\% of users are \textit{not}
willing to share any information about themselves with a company in exchange
for a free or subsidized app, despite ad-driven free apps being common on
mobile app marketplaces. Smartphone privacy receives significant attention
from both researchers and developers.  Ironically, we believe that while
privacy protection is a worthwhile goal, privacy itself may be part of the
smartphone privacy problem.

The reason is that understanding privacy implications requires smartphone
users to answer difficult questions. If I install and use this app, what will it
be able to determine about me? How much of the data that this app is
collecting is really necessary? What are the privacy implications of even the
legitimate data that this app is collecting? Accurate answers to these
questions remain elusive at best. There are billions of dollars at stake for
companies in determining how to do more accurate mobile data analytics, and
few if any have a business interest in divulging either how their algorithms
work or what they really know about us. While new tools help smartphone users
determine how much data smartphone apps collect, what is done with the data
that is collected remains opaque.

Obviously users can always choose not to use apps with which they feel
uncomfortable.  There are many projects looking at how to make safer
app marketplaces by preventing malicious apps that \textit{only} want to
steal personal information.  However, this focus on malicious apps obscures a
harder truth, viz., even the legitimate data collected by non-malicious apps
creates a privacy risk for smartphone users. For a typical user who wants to
read email, browse the web, take pictures, and use social networking
and messaging clients, legitimate data collection by legitimate apps still
constitutes a significant risk to their privacy. Even after preventing
unnecessary data collection by legitimate apps, a tradeoff between privacy
and usability remains. The only options left to smartphone users at this point
are to remove useful apps or to cease using their smartphone entirely -- both
unattractive.

To further investigate smartphone user thoughts on privacy, we distributed an
IRB-approved survey---asking about their thoughts on smartphones knowing their
address, friends, activity level, income and weight---to students, faculty and
staff of the University at Buffalo\footnote{We discuss this survey in more
detail in our concurrent publication submitted to HotPlanet 2014.}. No
incentives were provided for completing  the  survey,  and all  respondents
were  required to indicate consent before proceeding to the questions.  Over
four days, we recorded 91 responses. First,  we  found  that respondents  to be
reasonably suspicious of what apps might know about them, with 52\% indicating
that  an app  might  know  at  least  one personal attribute  to a  level  that
we marked  as  unreasonable  today---such as income and weight---but  only 18\%
indicating that  apps might know two attributes of unreasonable levels.  And
when asked about mocking,  of the 91 users that  completed  the  survey, 82\%
wanted  to mock  at least one attribute and 60\% wanted to mock two, with
mocking users requesting an average of 2.6 mocking attributes each. Our results
show that users are concerned with what their smartphone may know about them,
and users are interested in the ability to better control their private data on
their devices.

It is at that point that mocking has a role to play. In contrast to the
uncertainties caused by privacy, mocking provides control. Instead of
wondering what information an employer required app collects on a
bring-your-own-device, users can set up mocking objectives that
ensure that it appears that they work regular hours. Instead of wondering
which apps might be collecting information about their drinking habits, users
can set up mocking objectives to conceal their visits to bars. We believe
that this type of control over their digital identities will be appealing
to users, since it is similar to the control they already have when
interacting with other online services. While Facebook users know that
Facebook is collecting data about them, they also exercise control over the
impression they create. \PocketMocker{} provides smartphone users with the
same control over their smartphone-derived digital identities.


\subsection{Mocking Scenarios and Types}

Consider these four scenarios:

\begin{itemize}

\item Bob wants to appear more active. On Monday he takes a walk to get some
exercise. The next day, he doesn't take a real walk, but while he is sitting
at his desk his smartphone mocks another walk.

\item Alice wants to appear more healthy, but on Monday she visits a fast
food restaurant where she enjoys an unwholesome meal. As she eats, however, her
smartphone mocks a visit to a nearby organic salad delicatessen.

\item Teenager Jerry's parents use a smartphone app to monitor his late-night
  ventures and to ensure that he returns home by an imposed curfew. One
  night Jerry remains out on the town later-than-allowed with his friends,
  but his phone has already mocked him dutifully returning home on time.

\item Carol's employer uses her phone to monitor her attendance. During the
workday, she surreptitiously slips out for a latte with a friend.  Meanwhile,
her phone records her apparent continued presence at her desk.

\end{itemize}

These examples highlight the difference between \textit{privacy} and
\textit{mocking}. None of our characters' objectives can be accomplished
through privacy, since in each case achieving the objective requires using
data to manipulate an app. Of course, Jerry could remove the app that his
parents installed, as could Carol, but his parents and her employer would
likely notice. In the cases of Alice and Bob, we can assume that
they have been asked or incentivized to install these health-monitoring apps
but may not feel fully-comfortable with their operation. Similarly, however,
removing or disabling the apps may not be an attractive option. 

These scenarios also illustrate examples of two different types of
mocking: record-and-replay and time shifting.

\subsubsection{Record and peplay mocking:\space}

In record-and-replay mocking the objective can be directly embedded in the
mocked activity. Bob's recorded walk inherently makes him appear more active
the more he replays it, and Alice's visit to the healthy restaurant makes her
appear more healthy. In other cases, the objective can also be to obscure
another activity, replacing something undesirable with something desirable.
As Alice replays the desirable visit to the healthy restaurant, it provides
cover for her undesirable visit to the unhealthy restaurant.

In other cases, \PocketMocker{} users may simply want to have their smartphones
mock staying in one place while they in fact go to another, as in
Carol's example when she visits with her friend instead of working. While this
behavior is similar to that offered by traditional record-and-replay systems,
this variant requires \PocketMocker{} be able to extend or loop a recorded
session for an indefinite amount of time in order to mask an activity and to
ensure time continuity between the mocking trace and the user's real activity.

\subsubsection{Time shift mocking:\space}

In time-shift mocking, a user wants it to appear that something that happened
at one time actually happened at another. Time shifting can move an activity
later or earlier, the later exemplified by Jerry's example where he wants his
parents to think that he arrived home punctually. We call the former
\textit{forward} time shifting and the latter \textit{backward}.

While conceptually similar, forward and backward time shifting create
different design requirements. Backward time shifting---making something that
happens in the future appear to happen earlier---requires either being able
to synthesize a transition from the current context to the mocked context or
having a pre-recorded trace that accomplishes the same thing. In Jerry's
example, in order to look like he returned home early, his smartphone must
either be able to create a location transition from his current location to
home, or he must have previously recorded this transition.

Forward time shifting is somewhat easier, since the user's real transition can
be recorded, saved, and then held until the user is ready to replay it at a
later point in time. If Carol wants to leave work early, she can record her
transition to home but then delay it for several hours until the working day
is over. Like record-and-replay, however, this type of mocking also requires
\PocketMocker{} to be able to dwell in a particular context while the user
moves to another by looping a portion of a pre-recorded context.
