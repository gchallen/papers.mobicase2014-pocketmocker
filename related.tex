\section{Related Work}
\label{sec-related}

While \PocketMocker{} is the first system to provide objective-driven context
mocking, worries about smartphone app data collection have led to a number of
previous efforts in this area. First, we focus on the  approaches similar to
\PocketMocker{} which deny apps access to data by returning faked data. Then,
we look at novel ways to detect malicious apps or malicious app behavior
through static analsysis and information flow tracking. Finally, we examine
proposals to improve the Android's permission model to make it more effective
and user-friendly.

% 02 Apr 2014 : GWA : NVD TODO : First pass. Target is 1 page. Can borrow
% from: AppFence, BlueSeal tech report (blueseal.cse.buffalo.edu), TaintDroid
% (although a bit dated)

% 02 Apr 2014 : GWA : Start with the stuff most similar to mocking.

\subsection{Mocking Approaches}

AppFence~\cite{droids-ccs11} is one example that uses data shadowing,
blocking, and mocking to selectively deny data to apps on a per-permission
basis. When apps request data the user has chosen to deny them, such as their
phone number or email address, empty or fixed bogus values are returned.
MockDroid~\cite{mockdroid-hotmobile11} is another example of a similar
system. Both these approaches focus on achieving privacy by limiting access
to data, rather than achieving user objectives by manipulating data that apps
do have access to, and we consider these efforts orthogonal.

% 04 Apr 2014 : GWA : This is quite related... use the stuff from Taeyeon's
% talk.

\subsection{Record and Replay}

\PocketMocker{} utilizes the idea of record and replay as a first attempt of
implementing objective-driven mocking on Android. Record and replay is not a
new concept in general, but there has recently been a focus on instrumenting
this type of procedure on mobile devices~\cite{recordreplay-hotmobile11}.
There are a few systems in existence, but the majority of research efforts
for this type of system seems to be focused on developer testing and app
performance. For instance, there is RERAN~\cite{gomez2013reran}, which
captures input events at the filesystem level for later programmatic replay;
another example of such a testing system is
VanarSera~\cite{vanarsena-mobisys14}, which records event data then
distributes it to multiple ``monkeys'' or slave devices for parallel testing;
there is also AppInsight~\cite{appinsight-osdi12}, an app-layer
instrumentation that helps identify critical paths in per-app user
transactions. Despite this focus, there is interest in using record and
replay systems to bring permission issues to light~\cite{permissions-spsm12}.

% 02 Apr 2014 : GWA : Cover some attempts to improve Android's permission
% model. I think that there has been some work on incremental permissions,
% probably other stuff.
% NVD: AdDroid: Privilege Separation for Applications andAdvertisers in Android
% NVD: “These Aren’t the Droids You’re Looking For” Retrofitting Android to 
%       Protect Data from Imperious Applications
% NVD: Flow Permissions for Android

\subsection{Permissions Improvements}

A number of previous systems have explored ways to overcome Android's
``take-it-or-leave-it'' permission model by allowing apps to run but
selectively denying them access to information that they think they can
access. There are other approaches out there to try to amend this model by
allowing users to decide which permissions to accept~\cite{apex-asiaccs10},
but users could accidentally reduce app functionality by rejecting a
particular permission. Giving users the ability to selectively accept
permissions does not fix the current permissions model because many
apps request permissions for private data that are necessary only to
third-party components, like advertising libraries~\cite{addroid-asiaccs12}.
To thwart this, AdDroid separates the requested permissions by
advertiser-required and application-required.
