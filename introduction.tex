% 25 Mar 2014 : GWA : Goal is 1 page including bullet contributions at end and
% paper structure paragraph.

\section{Introduction} \label{sec-introduction}

With the digital portraits painted by smartphones becoming ever clearer, we
believe it is time to give users more control over their smartphone-derived
digital identities. Many other components of our digital lives already provide
ways to curate the information we provide in order to mold our digital
personas. Users of dating sites post pictures that they hope others will find
attractive, users of photo-sharing sites post more photos of themselves out
performing interesting activities than of mundane sedentary ones, and users of
social networking sites carefully choose the movies and music they list in
their profiles to create a particular desired impression. While using these
online services requires surrendering some amount of privacy, the active
participation they require provides users with some control over what
information they reveal and the impression they create. In contrast, the
passive data collection that smartphones facilitate represents a dangerous
simultaneous loss of \textit{both} privacy and control.

% 25 Mar 2014 : GWA : TODO : Look at Android Ap Ops
% http://www.google.com/search?q=android+selective+permissions. Do similar
% mechanisms exist on other platforms? Try to bring this paragraph up to
% date...

Today's smartphone platforms have attempted to address this loss of privacy
through permission mechanisms that limit apps' access to user information.
Unfortunately, this approach has many well-documented problems: apps request
permissions they don't need~\cite{taintdroid-osdi,demystified-ccs11}, and users
do not understand the implications of permissions that apps
request~\cite{androidperms-soups12}. The ``take-it-or-leave-it'' model used by
Android provides no options for users uncomfortable with the permissions an app
requests other than not to install it. Even a more selective
``take-it-or-break-it'' approach~\cite{apex-asiaccs10} that could allow users
selectively to enable individual permissions is no cure.  Users are still
forced to make poorly informed tradeoffs between privacy and functionality, as
it is unclear how an app might behave if denied access to information or fed
random ersatz values. The result in practice is that users tend to give apps
the permissions they request.

% 25 Mar 2014 : GWA : TODO : Look at the flow permissions paper for some
% citation inspiration here.

One fundamental problem with all permission-based approaches to protecting
privacy on smartphones is that many apps legitimately require access to certain
kinds of sensitive information to function properly. When I am lost, my mapping
app must know where I am in order to route me to my destination. When I am
monitoring my fitness, my pedometer must be able to access the accelerometer to
count the number of steps I take each day. It is unreasonable to expect users
to be able to distinguish between legitimate and illegitimate information
requests, and burdensome and error-prone to ask them to enable data sources
only when they feel comfortable with what a particular app is doing.

Our solution takes a different approach. Instead of focusing on privacy by
limiting data collection, we aim to improve control by generating synthetic or
``mocked'' data to manipulate data-driven analytics as directed by the user, an
approach we call \textit{objective-driven context mocking}. In contrast to
privacy, which aims to limit access to data, mocking reduces the power of
legitimate data by injecting enough mocked data to achieve user-defined
objectives.  Unlike privacy, which requires hiding data and thus potentially
impacting apps' functionality, mocking ensures that apps continue to function
normally during each mocking session, making it simpler for users to understand
and use.

Our paper makes the following contributions:

\begin{enumerate}

\item We introduce objective-driven context mocking, a new approach to
protecting smartphone users' personal data that is orthogonal to privacy, and
use several examples to illustrate the power of our approach.

\item We describe the design of \PocketMocker{}, a system enabling
objective-driven context mocking -- that is, a system driven by users'
objectives in crafting their digital identities.  \PocketMocker{} mocks context
by capturing and replaying all information that could allow apps to pierce the
mocking environment: location, network characteristics, and sensor data.
\PocketMocker{}'s implementation consists of both Android platform
modifications that allows mocked data to be fed to unsuspecting apps and an app
that controls the mocking process by allowing users to record and replay
mocking traces.

\item We evaluate \PocketMocker{} and show it to be both desirable and
effective. Field testing of a \PocketMocker{} prototype demonstrates that it
can successfully mock several popular Android apps, and users.

\end{enumerate}

The rest of our paper is structured as follows. After motivating
\PocketMocker{} with several examples in Section~\ref{sec-motivation}, we
describe \PocketMocker{}'s design and implementation in
Sections~\ref{sec-design}~and~\ref{sec-implementation}.
Section~\ref{sec-evaluation} evaluates our \PocketMocker{} prototype, showing
both that \PocketMocker{} works and that smartphone users are interested in
using the capabilities it provides. We review related work in
Section~\ref{sec-related}, discuss future plans in Section~\ref{sec-future},
and conclude in Section~\ref{sec-conclusion}.
