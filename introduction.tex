% 25 Mar 2014 : GWA : Goal is 1 page including bullet contributions at end and
% paper structure paragraph.

\section{Introduction} \label{sec-introduction}

With the digital portraits painted by smartphones becoming ever clearer, we
believe it is time to give users more control over their smartphone-derived
digital identities. Many other components of our digital lives already provide
ways to curate the information we provide in order to mold our digital
personas: users on social networking sites can already make themselves look
more attractive or seem more interesting by actively selecting the pictures
they upload or the activities they share. While using these online services
requires surrendering some amount of privacy, the active participation they
require provides users with some control over what information they reveal and
the impression they create. In contrast, the passive data collection that
smartphones facilitate represents a dangerous simultaneous loss of
\textit{both} privacy and control.

% 25 Mar 2014 : GWA : TODO : Look at Android Ap Ops
% http://www.google.com/search?q=android+selective+permissions. Do similar
% mechanisms exist on other platforms? Try to bring this paragraph up to
% date...

This loss of privacy is not ignored by smartphone platforms. These platforms
try to protect users by having apps request sensitive user data through
permission mechanisms. Unfortunately, the currently used model in Android---
where users have to either install the app with all of its permissions or not
install at all---leave users exposed to two problems: apps tend to request more
than required~\cite{taintdroid-osdi,demystified-ccs11} and users do not
understand why apps request particular permissions~\cite{androidperms-soups12}.
An alternative approach~\cite{apex-asiaccs10} allows users to decide which
permissions to grant, but users have to choose between app functionality and
their privacy because they are unaware how apps may behave if a particular
permission request is rejected. This leads to users accepting all permissions
for the apps they wish to install despite not fully knowing how that data may
be used. Overall, permissions are not user-centric as apps request access to
data for both legitimate and illegitimate actions: a navigation app may use
location data for both navigating and for advertising. It is unreasonable to
expect users to distinguish between legitimate and illegitimate information
requests, and burdensome and error-prone to ask them to enable data sources
only when they feel comfortable with what an app is doing.

% 25 Mar 2014 : GWA : TODO : Look at the flow permissions paper for some
% citation inspiration here.

Our solution takes a different approach. Instead of focusing on privacy by
limiting data collection, we aim to improve control by generating synthetic or
``mocked'' data to manipulate data-driven analytics as directed by the user, an
approach we call \textit{objective-driven context mocking}. In contrast to
privacy, which aims to limit access to data, mocking reduces the power of
legitimate data by injecting enough mocked data to achieve user-defined
objectives.  Unlike privacy, which requires hiding data and thus potentially
impacting apps' functionality, mocking ensures that apps continue to function
normally during each mocking session, making it simpler for users to understand
and use.

Our paper makes the following contributions:

\begin{enumerate}

\item We introduce objective-driven context mocking, a new approach to
protecting smartphone users' personal data that is orthogonal to privacy, and
use several examples to illustrate the power of our approach.

\item We describe the design of \PocketMocker{}, a system enabling
objective-driven context mocking. \PocketMocker{}'s implementation consists
of both Android platform modifications that allows mocked data to be fed to
unsuspecting apps and an app that controls the mocking process.

\item We evaluate \PocketMocker{} and show it to be both desirable and
effective. Field testing of a \PocketMocker{} prototype demonstrates that it
can successfully mock several popular apps and users are interested in using
it.

\end{enumerate}

The rest of our paper is structured as follows. After motivating
\PocketMocker{} with several examples in Section~\ref{sec-motivation}, we
describe \PocketMocker{}'s design and implementation in
Sections~\ref{sec-design}~and~\ref{sec-implementation}.
Section~\ref{sec-evaluation} evaluates our \PocketMocker{} prototype, showing
both that \PocketMocker{} works and that smartphone users are interested in
using the capabilities it provides. We review related work in
Section~\ref{sec-related}, discuss future plans in Section~\ref{sec-future},
and conclude in Section~\ref{sec-conclusion}.
